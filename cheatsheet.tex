\documentclass[10pt]{scrartcl}

\usepackage[utf8]{inputenc} % Required for inputting international characters
\usepackage[T1]{fontenc} % Output font encoding for international characters

\usepackage[margin=0pt, landscape]{geometry} % Page margins and orientation

\usepackage{graphicx} % Required for including images
\usepackage[most]{tcolorbox}
%\usepackage[usenames]{xcolor} % Required for color customization
\definecolor{mygray}{gray}{.75} % Custom color

\usepackage{url} % Required for the \url command to easily display URLs

\usepackage[ % This block contains information used to annotate the PDF
colorlinks=false, 
pdftitle={Timewarrior Cheatsheet}, 
pdfauthor={Michael Kluge}, 
pdfsubject={timew command overview}, 
pdfkeywords={Timewarrior, Cheatsheet}
]{hyperref}

\setlength{\unitlength}{1mm} % Set the length that numerical units are measured in
\setlength{\parindent}{0pt} % Stop paragraph indentation

\definecolor{cmdcolor}{HTML}{B7ACE0}
\definecolor{hintscolor}{HTML}{FFD7BA}
\definecolor{syntaxcolor}{HTML}{EDF2B1}

\renewcommand{\dots}{\ \dotfill{}\ } % Fills in the right amount of dots
\newcommand{\command}[2]{\texttt{#1}~\dotfill{}~#2\\} % Custom command for adding a shorcut
\newcommand{\sectiontitle}[1]{\paragraph{#1} \ \\} % Custom command for subsection titles

\newenvironment{cssec}[1]{%
\vspace*{-0.2cm}
\begin{tcolorbox}[colback= #1 , coltext=black, box align=top, size=minimal, no shadow, left=2mm,right=2mm]
\vspace*{0.1cm}
}
{
\vspace*{-0.2cm}
\end{tcolorbox}
\vspace*{0.2cm}
}

%----------------------------------------------------------------------------------------

\begin{document}

\begin{picture}(297,210) % Create a container for the page content

%----------------------------------------------------------------------------------------
%	TITLE SECTION 
%----------------------------------------------------------------------------------------

\put(10,205){ % Position on the page to put the title
\begin{minipage}[t]{210mm} % The size and alignment of the title
\colorbox{black}{\color{white}\Large Time Warrior Cheatsheet - call \texttt{timew <command> <options>}}
\hspace{0.1cm}
\colorbox{cmdcolor}{\color{black}\Large \strut Commands}
\hspace{0.1cm}
\colorbox{hintscolor}{\color{black}\Large \strut Hints}
\hspace{0.1cm}
\colorbox{syntaxcolor}{\color{black}\Large \strut Syntax}
\end{minipage}
}


%----------------------------------------------------------------------------------------
%	FIRST COLUMN SPECIFICATION
%----------------------------------------------------------------------------------------

\put(10,195){ % Divide the page
\begin{minipage}[t]{133mm} % Create a box to house text

%----------------------------------------------------------------------------------------
%	HEADING ONE
%----------------------------------------------------------------------------------------


\begin{cssec}{cmdcolor}
\sectiontitle{Start/Stop}
\command{start [<date>] [<tag> ...]}{start}
\command{track <interval> [<tag> ...]}{like start but closed}
\command{stop [<date>] [<tag> ...]}{stop}
\command{continue}{continues the last interval}
\command{cancel}{drops the current interval}
\end{cssec}

\begin{cssec}{cmdcolor}
\sectiontitle{Edit Records}
\command{delete @<id>}{deletes an interval}
\command{shorten  @<id> [@<id> ...] <duration>}{removes time from end of interval}
\command{lengthen @<id> [@<id> ...] <duration>}{add time to end of interval}
\command{move @<id> <date>}{move interval to new start time}
\command{split @<id> [@<id> ...]}{split into two equal intervals}
\command{join @<id>  @<id>}{joins two intervals to one}
\end{cssec}

\begin{cssec}{cmdcolor}
\sectiontitle{Tags}
\command{tag @<id> [<tag> ...]}{add tag to id}
\command{untag @<id> [<tag> ...]}{remove tag from id}
\command{tags}{list all tags ever used}
\end{cssec}

\begin{cssec}{cmdcolor}
\sectiontitle{Requests}
\command{summary [<interval>] [<tag> ...]}{summary for time and tags}
\command{day}{summary for the day}
\command{week}{summary for the week}
\command{gaps}{list all time gaps}
\end{cssec}

\begin{cssec}{cmdcolor}
\sectiontitle{Misc}
\command{show}{shows all config variables}
\command{extensions}{directory that contains the extensions}
\command{diagnostics}{show diagnostics information}
\command{export [<interval>] [<tag> ...]}{expors matching intervals as JSON}
\command{config [<name> [<value> | '']]}{set config variable to new value}
\end{cssec}

\begin{cssec}{hintscolor}
\sectiontitle{Control Hints}
\command{:quiet}{Turns off all feedback. For automation}
\command{:debug}{Runs in debug mode, shows many runtime details}
\command{:yes}{Overrides confirmation by answering 'yes' to the questions}
\command{:color}{Force color on, even if not connected to a TTY}
\command{:nocolor}{Force color off, even if connected to a TTY}
\command{:blank}{Leaves tracked time out of a report}
\command{:fill}{Expand time to fill surrounding available gap}
\command{:adjust}{Automatically correct overlaps}
\command{:ids}{Displays interval ID numbers in the summary report}
\end{cssec}

\footnotesize{Created by Michael Kluge, 2017; Released under the MIT license.}
\end{minipage} % End the first column of text
}

%----------------------------------------------------------------------------------------
%	SECOND COLUMN SPECIFICATION 
%----------------------------------------------------------------------------------------

\put(153,195){ % Divide the page
\begin{minipage}[t]{133mm} % Create a box to house text

\begin{cssec}{hintscolor}
\sectiontitle{Time Range Hints}
\command{:yesterday}{The 24 hours of the previous day}
\command{:day}{The 24 hours of the current day}
\command{:week}{This week}
\command{:month}{This month}
\command{:quarter}{This quarter}
\command{:year}{This year}
\command{:lastweek}{Last week}
\command{:lastmonth}{Last month}
\command{:lastquarter}{Last quarter}
\command{:lastyear}{Last year}
\end{cssec}

\begin{cssec}{syntaxcolor}
\sectiontitle{Time Interval Syntax}
\texttt{[from] <date>} \\
\texttt{[from] <date> to/- <date>} \\
\texttt{[from] <date> for <duration>} \\
\texttt{<duration> before/after <date>} \\
\texttt{<duration> ago} \\
\texttt{[for] <duration>} \\
\end{cssec}

\begin{cssec}{syntaxcolor}
\sectiontitle{Date and Time Syntax}
\command{<extended-date> [T <extended-time>]}{Extended date, optional extended time}
\command{<date> [T <time>]}{Date, optional time}
\command{<extended-time>}{Extended time}
\command{<time>}{Time}
\textit{extended-date}\\
\command{YYYY-MM-DD}{Year, month, day}
\command{YYYY-MM}{Year, month, 1st}
\command{YYYY-DDD}{Year, Julian day 001-366}
\command{YYYY-WwwD}{Year, week number, day number}
\command{YYYY-Www}{Year, week number, day 1}
\textit{extended-time}\\
\command{hh:mm[:ss]Z}{Hours, minutes, optional seconds, UTC}
\command{hh:mm[:ss][+/-hh:mm]}{Hours, minutes, optional seconds, TZ}
\textit{date}\\
\command{YYYYMMDD}{Year, month, day}
\command{YYYYWww}{Year, week number, day number}
\command{YYYYDDD}{Year, Julian day 001-366}
\textit{time}\\
\command{hhmm[ss]Z}{Hour, minutes, optional seconds, UTC}
\command{hhmm[ss][+/-hh[mm]]}{Hour, minutes, optional seconds, TZ}
\end{cssec}

\begin{cssec}{syntaxcolor}
\sectiontitle{Durations}
\texttt{'P' [nn 'Y'] [nn 'M'] [nn 'D'] ['T' [nn 'H'] [nn 'M'] [nn 'S']]} \\
\texttt{days, day, d, hours, hour, hrs, hr, h, minutes, ... lots more}\\
\end{cssec}


%----------------------------------------------------------------------------------------

\end{minipage} % End the third column of text
} % End the third division of the page
\end{picture} % End the container for the entire page

%----------------------------------------------------------------------------------------

\end{document}